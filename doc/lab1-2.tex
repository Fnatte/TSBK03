% -*- compile-command: "pdflatex lab1-2.tex"; -*-
\documentclass[a4paper,12pt]{article}
\usepackage[utf8]{inputenc}
\usepackage[english]{babel}
\usepackage{listings}
\usepackage{hyperref}

\begin{document}

\section{Bump mapping}

\begin{itemize}
  \item Vilken bumpmappning tycker du är att föredra, vykoordinater eller texturkoordinater? Varför? Vad är skillnaden mellan att arbeta i vy- och texturkoordinater? Vilken bumpmappning (2a eller 2b) är lämplig för normalmapping?

    Vi tycker att bumpmapping 2b är att föredra då man per frame kan kalkylera transformationsmatrisen från vy till texturkoordinater mvt och på så sätt undvika onödiga beräkningar.

  \item Definierar du bumpmappen som avstånd in i eller ut ur objektet? Var spelar det in?

    Bumpmapen går både in och ut då grundnivån i bumpmapen är halvvägs mellan vitt och svart.

  \item Blev Mvt rätt med mat3 i 2b? Om inte, vad gjorde du åt det?

    Vi transponerade den, det blev bra då.

\end{itemize}


\end{document}