% -*- compile-command: "pdflatex lab2.tex"; -*-
\documentclass[a4paper,12pt]{article}
\usepackage[utf8]{inputenc}
\usepackage[english]{babel}
\usepackage{listings}
\usepackage{hyperref}

\begin{document}

\section{Skinning}

\begin{itemize}
\item  I del 2, translaterade du med translationsmatris eller med addition på koordinater? Spelar det någon roll?

  Translationsmatris skapade vi en.

\item  Hur löste du GPU-skinningen? Vilken information behövde skickas till shadern?

  Vi behövde ladda upp enbart matriserna boneTransforms som vi kallade den, en matris per ben. I boken kallas den matrisen M, vilket ypperligt namn.

\item  Om man gör skinning med en mer komplex modell (armar och ben mm), behövs vilolägesrotationer egentligen?

  Man kan tänka sig att fötterna på en person är lätt roterade i viloläge, men vi skulle lika gärna kunna skippa det enligt vår åsikt.

\end{itemize}

\end{document}